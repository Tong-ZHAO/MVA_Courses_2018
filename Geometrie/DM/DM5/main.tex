\documentclass[11pt]{article}

    \usepackage{geometry, amsmath, amsthm, latexsym, amssymb, graphicx}
    \usepackage{dsfont}
    \usepackage{mathrsfs}
    \usepackage[utf8]{inputenc}
    \usepackage[numbered,framed]{matlab-prettifier}
    \usepackage{filecontents}
    \usepackage[francais,english]{babel}
    \usepackage{setspace}
    \usepackage{verbatim}
    \usepackage{mathrsfs}

    \renewcommand{\baselinestretch}{1.8} 

    \geometry{margin=1in, headsep=0.25in}
    \graphicspath{{./imgs/}}
    
    \parindent 0in
    \parskip 12pt
    
    \begin{document}
    
    \title{HWK5 - Géométrie et espaces de formes}
    
    \thispagestyle{empty}
    
    \begin{center}
    {\LARGE \bf Géométrie et Espaces de Formes - Exercice 5}\\
    \vspace{1em}
    {\large Tong ZHAO (tong.zhao@eleves.enpc.fr)}\\
    \end{center}

    \textbf{Exo 5.1} 

    Etant donnée que $f \in \mathscr{C}_{loc}^{1, lip} (E, F)$, il existe $K_R \ge 0$ tel que pour tous $x,y$ dans la boule
    $B_E(0, R)$ sur $E$, on a $|df(x) - df(y)| \le K_R|x-y|$ et $|f(x) - f(y)| \le K_R'|x-y|$.

    Etant donnée que $g \in \mathscr{C}_{loc}^{1, lip} (F, G)$, il existe $K_R \ge 0$ tel que pour tous $x,y$ dans la boule
    $B_F(0, R)$ sur $F$, on a $|dg(x) - dg(y)| \le K_R|x-y|$.

    Donc on a:

    \vspace{-5em}
    \begin{align*} 
      |dg(f(x))df(x) - dg(f(y))dg(y)| &= |dg(f(x))df(x) - dg(f(y))df(x) + dg(f(y))df(x) - dg(f(y))df(y)| \\
      &\le |df(x)| |dg(f(x)) - dg(f(y))| + |dg(f(y))| |df(x) - df(y)| \\
      &\le K_G |df(x)| |f(x) - f(y)| + K_F |dg(f(y))| |x-y| \\
      &\le K_G |df(x)| K_G' |x-y| + K_F |dg(f(y))| |x-y| \\
      &= (K_G K_G' |df(x)| + K_F |dg(f(y))|) |x-y| \\
      &= K |x-y|
    \end{align*}
    \vspace{-4em}

    Sachant que $g \circ f \in \mathscr{C}^1 (E, G)$, $g \circ f \in \mathscr{C}^{1,lip}_{loc}$.







    \textbf{Exo 5.2}

    (1) Selon la formule trouvée dans le section 1,

    \vspace{-5em}
    \begin{align*} 
      H_r(q,p) &= \frac{1}{2} \left \langle Kj, j \right \rangle \\
      &= \frac{1}{2} \left \langle K\sum_{i=1}^k \delta_{x_i}^{p_i}, \sum_{j=1}^k \delta_{x_j}^{p_j} \right \rangle  \\
      &= \frac{1}{2} \sum_{i, j = 1}^k p_i K(x_i, x_j) p_j
    \end{align*}
    \vspace{-3em}

    (2) Si $V$ s'injecte continuement dans $C^2_b (\mathbb{R}^d, \mathbb{R}^d)$, selon la proposition V.2, $\delta_x^\alpha$ est dans 
    $\mathscr{C}_{loc}^{1,lip} (\mathbb{R}^d \times \mathbb{R}^d, V^*)$. On en déduit donc que $K$ est $\mathscr{C}_{loc}^{1,lip} (\mathbb{R}^d \times \mathbb{R}^d, \mathbb{R})$. 

    (3) On calcule ses dérivées partielles de $H_r$ et on obtient les équations hamiltoniennes comme suit:

    \vspace{-2em}
    $$
    \begin{cases} 
    \dot{x_i} = \frac{\partial H_r}{\partial p_i} = \sum_{j=1}^k p_j K(x_i, x_j)\\
    \dot{p_i} = -\frac{\partial H_r}{\partial x_i} = -p_i \sum_{j=1}^k \frac{\partial K}{\partial x_i} (x_i, x_j) p_j
    \end{cases}
    $$
    \vspace{-2em}

    \textbf{Exo 5.3} 

    \vspace{-5em}
    \begin{align*} 
      &\big(dj(q+\Delta q, p + \Delta p) \cdot \big(\delta q, \delta p \big) | v\big) -  \big(dj(q, p) \cdot \big(\delta q, \delta p \big) | v\big) \\ 
      &= \big(\delta p, v \circ (q + \Delta q) \big) + \big (p + \Delta p | dv \circ (q + \Delta q) \cdot \delta q \big ) - \big(\delta p, v \circ q \big) - \big (p | dv \circ q \cdot \delta q \big ) \\
      &= \big(\delta p, v \circ (q + \Delta q - q) \big) + \big (dv^* \circ (q + \Delta q) \cdot (p + \delta p) - dv^* \circ q \cdot p | \delta q \big ) \\
      &\le K |v|_V |\Delta q| |\delta p| + K |v|_V |\Delta q| |\Delta p| |\delta q|
    \end{align*}
    \vspace{-3em}

    ce qui nous indique que $dj$ est localement lipschitzienne et donc $j \in \mathscr{C}_{loc}^{1,lip} (\mathscr{B} \times B_e^*, V^*).$

    \textbf{Exo 5.4}

    Sachant que $q_1(s)$ et $y(s)$ sont continue sur $[0,1]$, $s \to q_1(s) - y(s)$ est continue en $s$ pour tout $s \in [0,1]$.

    Selon le lemme V.1, $g = \int_0^1 |q_1(s) - y(s)|^2 ds$ est dans $\mathscr{C}^1$ et sa dérivée est $dg(q_1(s)) = q_1(s) - y(s)$, qui est dans 
    $\mathscr{C} ([0, 1], \mathbb{R}^d)$.


    \textbf{Exo 5.5}

    \vspace{-5em}
    \begin{align*} 
      H_r(q,p) &= \frac{1}{2} (Kj | j) \\
      &= \frac{1}{2} \left \langle \int_0^1 p(r) K(q(r), x) dr, \int_0^1 \delta_{x_s}^{p_s} ds  \right \rangle  \\
      &= \frac{1}{2} \int_0^1 \int_0^1 p^T (r) K(q(r), q(s)) p(s) dr ds 
    \end{align*}
    \vspace{-3em}

    On calcule ses dérivées partielles de $H_r$ et on obtient les équations hamiltoniennes comme suit:

    \vspace{-2em}
    $$
    \begin{cases} 
    \dot{q} (x) = \frac{\partial H_r}{\partial p} (x) = \int_0^1 K(x, q(s)) p(s) ds\\
    \dot{p} (x) = -\frac{\partial H_r}{\partial q} (x) = p(x)^T \int_0^1 \frac{\partial K}{\partial q} (q, q(s)) p(s) ds
    \end{cases}
    $$
    \vspace{-2em}


    \end{document}

    \textbf{Exo 4.1}

    Soit $H_f$ tient, on a pour tous $\epsilon_1, \epsilon_2 \in \mathbb{R}^+$ suffisament petits:

    \vspace{-5em}
    \begin{align*} 
      |f(q+\epsilon_1, u) - f(q,u)| &\le K\epsilon_1 (|u| + 1) \\
      |f(q, u+\epsilon_2) - f(q,u)| &\le K\epsilon_2 (|q-b|+1)
    \end{align*}
    \vspace{-4em}

    On pose $q' = q + \epsilon_1$, $u' = u + \epsilon_2$. Par l'inégalité triangulaire:

    \vspace{-5em}
    \begin{align*} 
      |f(q', u') - f(q,u)| &\le |f(q', u') - f(q,u')| + |f(q, u') - f(q,u)| \\ 
       &\le K(|u'| + 1)|q'-q| + K(|q-b|+1)|u'-u| \\
       &= K\Big(\big(|q-b|+1 \big)|u'-u| + \big(|u'|+1 \big) |q'-q| \Big)
    \end{align*}
    \vspace{-4em}

    ce qui entraîne la condition (33).

    \textbf{Exo 4.2}

    (1) On pose $\delta m = (\delta x_1, \cdots , \delta x_k)$ une petite perturbation en $m$, et donc:

    \vspace{-4em}
    \begin{align*} 
      \cfrac{\partial f}{\partial m} (m, u) &= \cfrac{f(m+\delta m, u ) - f(m,u)}{\delta m} \\
      &= \Big( \cfrac{u(x_1 + \delta x_1) - u(x_1)}{\delta x_1}, \cdots, \cfrac{u(x_k + \delta x_k) - u(x_k)}{\delta x_m}  \Big ) \\
      &= \Big ( du(x_1), \cdots, du(x_k) \Big )
    \end{align*}
    \vspace{-4em}

    Donc:

    \vspace{-5em}
    \begin{align*} 
      \cfrac{\partial f}{\partial m} (m, u) \delta m = \Big ( du(x_1)\delta x_1, \cdots, du(x_k)\delta x_k \Big )
    \end{align*}
    \vspace{-4em}

    Sachant que $u \in V$ admissible,

    \vspace{-4em}
    \begin{align*} 
      \cfrac{\partial f}{\partial u} (m, u) &= \cfrac{f(m, u + \delta u) - f(m,u)}{\delta u} \\
      &= \Big( \cfrac{(u+\delta u)(x_1 ) - u(x_1)}{\delta u}, \cdots, \cfrac{(u + \delta u)(x_k) - u(x_k)}{\delta u}  \Big ) \\
      &= \Big ( \cfrac{\delta u (x_1)}{\delta u}, \cdots, \cfrac{\delta u (x_k)}{\delta u} \Big )
    \end{align*}
    \vspace{-4em}

    Donc on a: $\frac{\partial f}{\partial u} (m, u) \delta u = \big(\delta u(x_1), \cdots, \delta u(x_k) \big)$.

    (2) Du fait que $u \in V$ est admissible, $du$ est continue sur $\mathscr{B}$, ce qui nous indique que $\frac{\partial f}{\partial m} (m, u) \delta m= \big (du(x_1) \delta x_1, \cdots, du(x_k) \delta x_k \big)$ est continue sur $(\mathbb{R}^d)^k$. 
    Du fait que la fonctionnel $\delta$ est linéaire, $\frac{\partial f}{\partial u} (m, u) \delta u = \big(\delta u(x_1), \cdots, \delta u (x_k) \big)$ est continue. On en déduit que $f$ est $\mathscr{C}^1$.

    \vspace{-4em}
    \begin{align*} 
      \Big|\frac{\partial f}{\partial m} (m, u) \delta m \Big| &\le \sum_{i=1}^k |du(x_i)\delta x_i| \\
      &\le C |u|_V |\delta m|
    \end{align*}
    \vspace{-4em}

    Donc on a: $|\frac{\partial f}{\partial m}| \le C|u|_V$.

    \vspace{-4em}
    \begin{align*} 
      \Big|\frac{\partial f}{\partial u} (m, u) \delta u \Big| &\le \sum_{i=1}^k |\delta u(x_i)| \\
      &\le C |\delta u|_V |m|
    \end{align*}
    \vspace{-4em}

    Donc on a: $|\frac{\partial f}{\partial u}| \le C|m|$. 

    En prennant $K = C$, $b = 0$, on retrouve la condition $(H_f)$.

    \textbf{Exo 4.3} 

    (1) On calcule tout d'abord la dérivée partielle par rapport à $q$:

    \vspace{-4em}
    \begin{align*} 
      \frac{\partial f}{\partial q} (q, u) &= \cfrac{u \circ (q + \delta q) - u \circ q}{\delta q} \\
      &= \cfrac{u \circ q + \delta q \cdot du \circ q + o(\delta q) - u \circ q}{\delta q} \\
      &= du \circ q
    \end{align*}
    \vspace{-4em}

    Par hypothèse, $u \in V$ est admissible donc $\frac{\partial f}{\partial q}$ est continue. On calcule ensuite la dérivée partielle par rapport à $u$:

    \vspace{-4em}
    \begin{align*} 
      \frac{\partial f}{\partial u} (q, u) &= \cfrac{(u + \delta u)\circ q - u \circ q}{\delta u} \\
      &= \cfrac{u \circ q + \delta u \circ q - u \circ q}{\delta u} \\
      &= \cfrac{\delta u \circ q}{\delta u}
    \end{align*}
    \vspace{-4em}

    Comme $\delta u$ est linéaire, $\frac{\partial f}{\partial u}$ est continue. On en déduit que $f$ est $\mathscr{C}^1$.

    (2) Inspiré par l'exercice précédente, on a:

    \vspace{-4em}
    \begin{align*} 
      \Big |\frac{\partial f}{\partial q} (q, u) \delta q \Big | &= |\delta q \cdot du \circ q| \\
      & \le C |u|_V |\delta q|
    \end{align*}
    \vspace{-4em}

    Et donc $\frac{\partial f}{\partial q} (q, u) \le C |u|_V$.

    \vspace{-4em}
    \begin{align*} 
      \Big |\frac{\partial f}{\partial u} (q, u) \delta u \Big | &= |\delta u \circ q| \\
      & \le C |\delta u|_V |q|
    \end{align*}
    \vspace{-4em}

    Et donc $\frac{\partial f}{\partial u} (q, u) \le C |q|$. On en déduit que $f$ satisfait les conditions $(H_f)$.

    \textbf{Exo 4.4} Du fait que $C(q,u) = \frac{1}{2} |u|^2_V$ est indépendant de $q$, la dérivée partielle de $C$ par rapport à $q$ est donc $0$, qui satisfait la seconde hypothèse de $(H_C)$.

    On calcule la dérivée partielle par rapport à $u$:

    \vspace{-4em}
    \begin{align*} 
      \frac{\partial C}{\partial u} (q, u) \delta u &= C(q, u + \delta u) - C(q, u) \\
      &= \frac{1}{2} |u + \delta u|_V^2 - \frac{1}{2} |u|_V^2 \\
      &= \left \langle u, \delta u \right \rangle_V + o(\delta u) 
    \end{align*}
    \vspace{-4em}

    Et donc:

    \vspace{-4em}
    \begin{align*} 
      \Big |\frac{\partial C}{\partial u} (q, u) \delta u\Big| &= |\left \langle u, \delta u \right \rangle_V | \\
      &\le |u|_V |\delta u|_V
    \end{align*}
    \vspace{-4em}

    En prennant $K = 1$, on a $|\frac{\partial C}{\partial u}| \le K|u|$, qui satisfait la première condition de $(H_C)$.

    \textbf{Exo 4.5} 

    (1) D'après le théorème IV.3, le hamiltonien $H(q, p, u) = \big (p | f(q, u) \big ) - C(q, u)$. En prenant le côut $C(q, u) = \frac{1}{2} |u|^2_V $ dans l'exercice précédente, on a:

    \vspace{-4em}
    \begin{align*} 
      H(q, p, u) &= \big ((p_1, \cdots, p_k )| (u(q_1), \cdots, u(q_k)) \big ) - \frac{1}{2} |u|_V^2 \\
      &= \sum_{i=1}^k (p_i | u(q_i)) - \frac{1}{2}|u|_V^2
    \end{align*}
    \vspace{-4em}

    Les équations hamiltoniennes sont:

    \vspace{-2em}
    $$
    \begin{cases} 
    \dot{q_i} = \frac{\partial H}{\partial p_i} = u(q_i)\\
    \dot{p_i} = -\frac{\partial H}{\partial q_i} = - \frac{\partial f}{\partial q_i}^\star (q_i, u_i) p_i\\
    \frac{\partial H}{\partial u} = \frac{\partial f}{\partial u}^\star p - u = 0
    \end{cases}
    $$
    \vspace{-2em}

    Et donc la solution $u^\star = \frac{\partial f}{\partial u}^\star p$

    (2) Comme précédemment, le hamiltonien est:

    \vspace{-4em}
    \begin{align*} 
      H(q, p, u) &= \big (p| q \circ u \big ) - \frac{1}{2} |u|_V^2
    \end{align*}
    \vspace{-4em}

    Les équations hamiltoniennes sont:

    \vspace{-2em}
    $$
    \begin{cases} 
    \dot{q} = \frac{\partial H}{\partial p} = u \circ q\\
    \dot{p} = -\frac{\partial H}{\partial q} = - \frac{\partial f}{\partial q_i}^\star (q_i, u_i) p_i = -(du \circ q)^\star p\\
    \frac{\partial H}{\partial u} = \frac{\partial f}{\partial u}^\star p - u = 0
    \end{cases}
    $$
    \vspace{-2em}

    Et donc la solution $u^\star = \frac{\partial f}{\partial u}^\star p$

    \end{document}
