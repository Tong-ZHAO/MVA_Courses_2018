\documentclass[11pt]{article}

    \usepackage{geometry, amsmath, amsthm, latexsym, amssymb, graphicx}
    \usepackage{dsfont}
    \usepackage{mathrsfs}
    \usepackage[utf8]{inputenc}
    \usepackage[numbered,framed]{matlab-prettifier}
    \usepackage{filecontents}
    \usepackage[francais,english]{babel}
    \usepackage{setspace}
    \usepackage{verbatim}
    \usepackage{mathrsfs}

    \renewcommand{\baselinestretch}{1.8} 

    \geometry{margin=1in, headsep=0.25in}
    \graphicspath{{./imgs/}}
    
    \parindent 0in
    \parskip 12pt
    
    \begin{document}
    
    \title{HWK5 - Géométrie et espaces de formes}
    
    \thispagestyle{empty}
    
    \begin{center}
    {\LARGE \bf Géométrie et Espaces de Formes - Exercice 6}\\
    \vspace{1em}
    {\large Tong ZHAO (tong.zhao@eleves.enpc.fr)}\\
    \end{center}

    \textbf{Exo 6.1}

    Soit $(\varphi_n)_{n \ge 0}$ une suite minimisante, on a: 

    \vspace{-3em} 
    $$\sup_n d_{G} (Id, \varphi_n) < \infty$$
    \vspace{-4em}

    Par suite $\exists\ (u_n) \in (L^2_V)^\mathbb{N}$ tel que $\varphi_n = \phi_1^{u_n}$
    et $\|u_n\|_2 = d_{G} (Id, \varphi_n)$. $\|u_n\|_2$ est borné en $n$. Il existe $u_{n_k}$ qui converge faiblement vers $u_\infty$.
    On a donc: 
    
    \vspace{-3em}
    $$\|u_\infty\|_2 \le \lim \|u_{n_k}\|_2$$
    \vspace{-4em}
    
    On a $\varphi_\infty \in G $ et $\varphi_\infty \overset{\underset{\triangle}{}}{=} \phi_1^\infty$, alors:
    
    \vspace{-3em}
    $$d_{G} (Id, \varphi_\infty) \le \|u_\infty\|_2 \le \lim d_{G}(Id, \varphi_{n_k})$$
    \vspace{-4em}

    Par la continuté faiblement du flot, $\varphi_{n_k}$ converge compacte uniformément vers $\varphi_\infty$
    d'où $E(\varphi_\infty) \le \lim E(\varphi_{n_k})$. $ d_{G} (\varphi, \varphi') \ge \|u_\infty\|_2 \ge \|u_\infty\|_1 \ge d_G(\varphi, \varphi')$.
    
    Ainsi:

    \vspace{-5em}
    \begin{align*}
      J(\varphi_\infty) &= R(d_{G}(Id, \varphi_\infty)) + E(\varphi_\infty) \\
      R(d_{G}(Id, \varphi_\infty)) &= \underline{\lim} R(d_{G} (Id, \varphi_\infty)) \\
      E(\varphi_\infty) &= \underline{\lim} E(\varphi_\infty)
    \end{align*}
    \vspace{-4em}

    donc $\inf J(\varphi) \le J(\varphi_\infty) = \underline{\lim} R(d_{G} (Id, \varphi_\infty)) + \underline{\lim} E(\varphi_\infty) \le \underline{\lim} J(\varphi_n) = \inf J(\varphi)$.

    On en déduit alors qu'il existe $\varphi^* \in G_V$ tel que $J(\varphi^*) = \inf_{\varphi \in G_V} J(\varphi)$.


\end{document}